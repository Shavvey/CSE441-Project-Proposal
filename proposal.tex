\documentclass[12pt,english]{article}
\usepackage{geometry}[margin=2]
\usepackage{blindtext}
\usepackage{enumerate}
\usepackage{parskip} 
\usepackage{siunitx}
\usepackage{amsmath}
\usepackage{amsfonts}
\usepackage{amssymb}
\usepackage{hyperref}
\usepackage{listings}
\usepackage{inconsolata}
\usepackage{parskip}
\usepackage{graphicx}
\usepackage{wrapfig}
\graphicspath{./images}
\author{
    Cole Johnson: \texttt{cole.johnson.@student.nmt.edu}
    \\ 
    John Runyon: \texttt{john.runyon@student.nmt.edu}
}
\title{
    CSE441: Project Proposal\\
    \large{Cryptanalysis of Hash Functions}
}
\begin{document}
\maketitle
\section*{Introduction and Outline of Project}
\subsection*{The Basics of Hash Functions}
Hashing, as an overall process, is a function that maps
plaintext data of any length into a fixed-length ciphertext
output--often called a digest. Hash functions, unlike encryption,
destroy information encoded in the plaintext, which means
the function is one-way and cannot be reversed to obtain
the plaintext again.

\begin{wrapfigure}{r}{0.5\textwidth}
    \begin{center}
      \includegraphics[clip=true,height=5cm, width=0.3\textwidth]{images/hash_function.png}
    \end{center}
    \caption{Basic diagram of a hash function}
  \end{wrapfigure}

Hash functions are a widely used type of cryptographic algorithm.
They can be used for a variety of purposes, such as
data integrity verification, password storage, and digital
fingerprint/signatures, and data indexing (often called hash-tables).
Since hash functions serve vital purposes in modern cryptography
and computer science, knowing the important mathematical properties
of a hash function (and how these can be implemented in programs)
is critical to understanding their function.

Once the function and real-world implementation of a
hash function is understood, it becomes clear why 
certain mathematical properties, such as determinism, 
pre-image resistance, and collision resistance are crucial. 
These properties ensure that hash functions can efficiently 
convert data of any size into a fixed-length output while 
preventing malicious actors from reversing or tampering with 
the data. The fixed length output is a major factor of any 
hashing algorithm - meaning that a user can input one character 
or one-hundred characters, and still receive a 256-bit hash as 
is the case with SHA-256.

\subsection*{Outline of Project}
Here is the sections of the project, along with a small description 
of what each of the sections will cover:
\begin{enumerate}[{\bf (a.)}]
    \item Introduction to Hash Functions
    \begin{enumerate}
        \item Basics of Hash Functions (including code examples, and diagrams)
        \item Legacy/Classic and Modern Hash Functions and their Applications
        \item Hash Functions Examples (MD5, SHA-256, Tornado)
    \end{enumerate} 
    \item Properties of Hash Functions
    \begin{enumerate}
        \item One-way Function (Pre-Image Resistance)
        \item Target Collision Resistance (2nd Pre-Image Resistance)
        \item Deterministic
        \item Avalanche Effect
        \item Computational Speed
    \end{enumerate}
    \item Cryptanalysis and Attacks on Cryptographic Hash Functions
    \begin{enumerate}
        \item Brute-Force Attacks
        \item One-way Function Inversion
        \item 2nd Pre-Image Resistance Attack
        \item Collision Attack
    \end{enumerate}
\end{enumerate}

\subsection*{Experimentation}
For our experimentation our group is wanting to look into 
the varying overhead required by different cryptographic 
algorithms such as MD5, SHA-1, and SHA-256. 
This could include using local machines for testing and online 
research to find and compare the available data on the performance 
of different hash functions in terms of speed and the use of 
resources. Our reasoning and conclusion of this experimentation 
would result in a conversation on the trade-offs between security 
and speed and the different use cases for each.

\subsection*{Conclusion}
Our project will explore the process of cryptoanalysis with various 
hash functions as mentioned above. By analyzing the performance, 
and security of varying hash functions we aim to 
better understand the trade-offs that exists amongst 
these algorithms. Our testing of these functions will 
give us a real world example of the hashes and the implications especially 
in terms of the overhead involved.

Hash functions are critical to the modern internet 
and computer infrastructure, so understanding their uses 
is not only important, but critical for security and ensuring 
data integrity (with uses of older hash functions as checksums). Our project will end with a review of the current state of hashing and what the future might look like in the world of post-quantum cryptography (PQC).
\end{document}